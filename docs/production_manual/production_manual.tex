\documentclass{article}

\usepackage[T1]{fontenc}
\usepackage{graphicx}
\usepackage[a4paper, margin=2cm]{geometry}
\usepackage{hyperref}
\usepackage{lastpage}
\usepackage{fancyhdr}
\usepackage{datetime}
\usepackage{glossaries}
\usepackage{xurl}

\title{Produktion Manual}

\begin{document}

\section*{Introduktion}

Dette dokument er en instuks i hvordan tracershop bruges i radiokemien.

\section*{Adgang til Tracershop}

For at tilgå Tracershop skal du først tildeles en CBAS brugergruppe,
mere specifikt, Tracershop-Prod eller Tracershop-Prod-admin.

Tracershop tilgåes fra en webbrowser på URL: https://tracershop.regionh.dk

Herefter vælges intern bruger.

\section*{Brug af Tracershop}

\subsection*{Favekoder I tracershop}

Tracershop anvender følgende farvekoder til ordre:
\begin{itemize}
  \item Rød - En kunde har lagt en ordre. Radiokemien har ikke accepteret at
  producerer denne ordre endnu.
  \item Gul - En accepteret ordre. Radiokemien har accepteret at producerer denne ordre
  \item Grøn - En Frigivet ordre. Orderen er blevet frigivet med hensyn til GMP.
  \item Sort - En afvis ordre. En ordre der er blevet afvist at blive produceret.
\end{itemize}

\subsection*{Ordre I tracershop}

Tracershop har 3 typer af ordre:

\begin{itemize}
  \item Aktivitets ordre - Et radioaktivt lægemiddel som bliver bestilt med en ønsket aktivitet.
  \item Injektions ordre - Et radioaktivt lægemiddel som bliver bestilt med en forudbestemt aktivitet per injektion.
  \item Isotop ordre - En ren isotop, som bestilles med en ønsket aktivitet
\end{itemize}

For at se ordre skal man trykke på Ordre i Navbaren. Hvor efter en kalender og
knapper for ugeplanen, injektions ordrene, og aktivitets sporestofferene, og
producerbare isotoper kommer frem. \\

Man kan klikke på knap for at se ordre med det produkt for en dato. Hvis man
ønsker at se andre ordre skal man klikke på den ønskede dato ude i kalenderen.

Kalenderen er farve kodet med den indre cirkel er mini

\end{document}